\documentclass{article}
% $Id: vim-dotfiles.tex,v 1.9 2004/09/28 18:19:11 nathan.neff@gmail.com Exp $
% Copyright 2009 Nathan Neff
% Taken and modified from the Donald Bindner:
% http://limestone.truman.edu/~dbindner/mirror/
% $Id: vi-ref.tex,v 1.9 2004/09/28 18:19:11 dbindner Exp $
% This program is free software; you can redistribute it and/or
% modify it under the terms of the GNU General Public License
% as published by the Free Software Foundation; either version 2
% of the License, or (at your option) any later version.
%
% This program is distributed in the hope that it will be useful,
% but WITHOUT ANY WARRANTY; without even the implied warranty of
% MERCHANTABILITY or FITNESS FOR A PARTICULAR PURPOSE.  See the
% GNU General Public License for more details.
%
% You should have received a copy of the GNU General Public License
% along with this program; if not, write to the Free Software
% Foundation, Inc., 59 Temple Place - Suite 330, Boston, MA  02111-1307, USA.

\usepackage{multicol}

\setlength{\textheight}{7.5 in}
\setlength{\textwidth}{10 in}
\setlength{\hoffset}{-2 in}
\setlength{\voffset}{-1 in}
\setlength{\footskip}{12 pt}
\setlength{\oddsidemargin}{1.5 in}
\setlength{\evensidemargin}{1.5 in}
\setlength{\topmargin}{.5 in}
\setlength{\headheight}{12 pt}
\setlength{\headsep}{0 in}

\setlength{\parindent}{0 in}

\ifx \pdfpagewidth \undefined
\else
 \pdfpagewidth=11in    % page width of PDF output
 \pdfpageheight=8.5in  % page height of PDF output
\fi

\begin{document}
\thispagestyle{empty}
\fontsize{9}{10}\selectfont

\newcommand{\key}[2]{#1 \hfill \texttt{#2}\par}
\newcommand{\head}[1]{{\large\textbf{#1}}\\}

\begin{multicols}{3}
{\Large Vi Reference Card}

\vskip 15pt

\vbox{\head{Modes}
These are my vim dotfiles.  There are many like it, but these are mine.
http://github.com/NathanNeff/vim-dotfiles

I use the comma as the <leader>.  If you use a different character as the
mapleader, then replace the comma with the character you use as the mapleader.
}
\newcommand{\leader}{,}

\vskip 10pt
\vbox{\head{File Operations}
\key{Save file, exit normal mode}{{\leader}f}
\key{Quit file/buffer, prompt if changed}{{\leader}d}
}

\vskip 10pt

\vbox{\head{Window Operations}
\key{Close all windows except current}{{\leader}1}}
\key{Close current window}{{\leader}0}}
\key{Split Window}{{\leader}2}}
\key{Split Window Vertically}{{\leader}3}}
}

\vskip 10pt

\vbox{\head{Search}
\key{Recursive file search (RGrep plugin)}{{\leader}sf}
\key{(Mnemonic: Search Files)}
\key{beginning of word, of blank delimited word}{b , B}
\key{end of word, of blank delimited word}{e , E}
\key{sentence back, forward}{( , )}
\key{paragraph back, forward}{$\{$ , $\}$}
\key{beginning, end of line}{0 , \$}
\key{beginning, end of file}{1G , G}
\key{line \texttt{\textit{n}}}{\textit{n}G\textrm{ or }:\textit{n}}
\key{forward, back to char \texttt{\textit{c}}}{f\textit{c} , F\textit{c}}
\key{forward, back to before char \texttt{\textit{c}}}{t\textit{c} , T\textit{c}}
\key{top, middle, bottom of screen}{H , M , L}
}

\vskip 10pt

\vbox{\head{Deleting text}
Almost all deletion commands are performed by typing \texttt{d}
followed by a \textit{motion}.  For example \texttt{dw} deletes a
word.  A few other deletions are:\par
\vskip 5pt
\key{character to right, left}{x , X}
\key{to end of line}{D}
\key{line}{dd}
\key{line}{:d}
}

\vskip 10pt

\vbox{\head{Yanking text}
Like deletion, almost all yank commands are performed by typing
\texttt{y} followed by a \textit{motion}.  For example \texttt{y\$}
yanks to the end of line.  Two other yank commands are:\par
\vskip 5pt
\key{line}{yy}
\key{line}{:y}
}

\vskip 10pt

\vbox{\head{Changing text}
The change command is a deletion command that leaves the editor in
insert mode.  It is performed by typing
\texttt{c} followed by a \textit{motion}.  For example \texttt{cw}
changes a word.  A few other change commands are:\par
\vskip 5pt
\key{to end of line}{C}
\key{line}{cc}
}

\vskip 10pt

\vbox{\head{Putting text}
\key{put after position or after line}{p}
\key{put before position or before line}{P}
}

\vskip 10pt

\vbox{\head{Registers}
Named registers may be specified before any deletion, change, yank, or
put command.  The general prefix has the form \texttt{"\textit{c}}
where \texttt{\textit{c}} may be any lower case letter.  For example,
\texttt{"adw} deletes a word into register \texttt{a}.  It may
thereafter be put back into the text with an appropriate put command,
for example \texttt{"ap}.\par
}

\vskip10pt

\vbox{\head{Markers}
Named markers may be set on any line of a file.  Any lower case letter
may be a marker name.  Markers may also be used as the limits for
ranges.\par
\vskip 5pt
\key{set marker \texttt{\textit{c}} on this line}{m\textit{c}}
\key{goto marker \texttt{\textit{c}}}{`\textit{c}}
\key{goto marker \texttt{\textit{c}} first non-blank}{'\textit{c}}
}

\vskip 10pt

\vbox{\head{Search for strings}
\key{search forward}{/\textit{string}}
\key{search backward}{?\textit{string}}
\key{repeat search in same, reverse direction}{n , N}
}

\vskip 10pt

\vbox{\head{Replace}
The search and replace function is accomplished with the \texttt{:s}
command.  It is commonly used in combination with ranges or
the \texttt{:g} command (below).\par
\vskip 5pt
\key{replace pattern with string}{:s/\textit{pattern}/\textit{string}/\textit{flags}}
\key{flags: all on each line, confirm each}{g , c}
\key{repeat last :s command}{\&}
}

\vskip 10pt
\vbox{\head{Regular expressions}
\key{any single character except newline}{.\textrm{ (dot)}}
\key{zero or more repeats}{*}
\key{any character in set}{[\dots]}
\key{any character not in set}{[\^\ \dots]}
\key{beginning, end of line}{\^\ , \$}
\key{beginning, end of word}{ $\backslash$< , $\backslash$>}
\key{grouping}{$\backslash(\ldots\backslash)$}
\key{contents of \texttt{\textit{n}}th grouping}{$\backslash$\textit{n}}
}

\vskip 10pt
\vbox{\head{Counts}
Nearly every command may be preceded by a number that specifies how
many times it is to be performed.  For example \texttt{5dw} will
delete 5 words and \texttt{3fe} will move the cursor forward to the
3rd occurance of the letter \texttt{e}.  Even insertions may be
repeated conveniently with this method, say to insert the same line
100 times.
}

\vskip 10pt

\vbox{\head{Ranges}
Ranges may precede most ``colon'' commands and cause them to be
executed on a line or lines.  For example \texttt{:3,7d} would delete lines
3$-$7.  Ranges are commonly combined with the \texttt{:s} command to
perform a replacement on several lines, as with
\texttt{:.,\$s/pattern/string/g} to make a replacement from the current line to
the end of the file.\par
\vskip 5pt
\key{lines \texttt{\textit{n-m}}}{:\texttt{\textit{n}},\texttt{\textit{m}}}
\key{current line}{:.}
\key{last line}{:\$}
\key{marker \texttt{\textit{c}}}{:'\textit{c}}
\key{all lines}{:\%}
\key{all matching lines}{:g/\textit{pattern}/}
}

\vskip 10pt

\vbox{\head{Files}
\key{write file (current file if no name given)}{:w \textit{file}}
\key{append file (current file if no name given)}{:w >>\textit{file}}
\key{read file after line}{:r \textit{file}}
\key{read program output}{:r !\textit{program}}
\key{next file}{:n}
\key{previous file}{:prev}
\key{edit new file}{:e \textit{file}}
\key{replace line with program output}{:.!\textit{program}}
}

\vskip 10pt

\vbox{\head{Other}
\key{toggle upper/lower case}{\~\ }
\key{join lines}{J}
\key{repeat last text-changing command}{.}
\key{undo last change, all changes on line}{u , U}
}

\end{multicols}

\vspace{\fill}
\copyright 2002-2005 Donald J.\ Bindner -- licensed under the terms of the GNU
General Public License 2.0 or later.
\end{document}

